

\section{Tablet}
\label{tablet}


\begin{table}[ht!]

	\begin{tabular}{r l|l p{12cm} }
		
		\textcolor{gray}{Especificação} &&& 	{Tablet Samsung - }\\
		\textcolor{gray}{Data} &&& 				{07/04/2014}\\
        \textcolor{gray}{Beneficiado} &&&		{Magazine Luiza} \\
        \textcolor{gray}{CNPJ} &&& 				{???} \\
        \textcolor{gray}{Número Nota} &&& 		{000.244.521} \\
		\textcolor{gray}{Quantidade} &&& 		{1} \\
		\textcolor{gray}{Valor} &&& 			{R\$1.583,12} \\
		\textcolor{gray}{Data Sheet} &&& 		{-} \\

		\textcolor{gray}{Função no projeto} &&& {O tablet será utilizado para a visualização da interface de usuário, que representa as informações do robô em um formato facilmente compreensível.} \\
		\textcolor{gray}{Razão da Escolha} &&& {O tipo de Tablet foi definido baseado nos requisitos de sistema operacional Android para funcionamento da interface desenvolvida. Dentre os tablets Androids, foi escolhido o que apresentou o menor preço dada a configuração mínima necessária.  

		\begin{itemize}
		  \item \textbf{...} 
		  \item ...
		  \item ...
		\end{itemize}}
		

	\end{tabular}
\end{table}

%\begin{figure}[h!]
 %\centering
 %\includegraphics[width=0.5\columnwidth]{Notebook/foto.pdf}
 %\caption{Notebook Dell}  
%\end{figure}